\section{Introduction}

Let $X$ be a vector space over a field $K \in \b{\R, \C}$.

\begin{definition}
  A map $f \col X \to \R$ is called a \emph{norm}, iff we have
  ~\begin{enumerate}
    \item If $x \in X \sm 0$, then $f(x) > 0$; and $f(0) = 0$.
    \item $f(kx) = \a{k} f(x)$ for all $k \in \R$, $x \in X$.
    \item $f(x+y) \le f(x)+f(y)$ for all $x, y \in X$.
  \end{enumerate}
\end{definition}

\begin{definition}
  A pair $\p{V, \n{\square}}$, where $\n{\square}$ is a norm on a vector space $V$, is called a \emph{normed space}.
  A normed space is \emph{Banach}, iff it is complete.
\end{definition}

\section{Completions}

\begin{definition}
  $\wh X$ is called a \emph{completion} of $X$, iff $\wh X$ is complete and $X$ is dense in $\wh X$.
\end{definition}

\begin{theorem}
  A completion $\wh X$ exists.
\end{theorem}

\begin{proof}
  Call two Cauchy sequences $\b{x_n}$ and $\b{y_n}$ equivalent, iff $\n{x_n-y_n} \to 0$, and let $\wh X$ be the resulting quotient.
  Since the point-wise sum of two Cauchy sequences is Cauchy, in this natural way we may introduce vector space structure on $\wh X$.
  The norm on $\wh X$ is introduced as
  $$ \n{\q{x_n}} := \lim_{n \to \infty} \n{x_n}. $$
  This map is defined correctly:
  \begin{enumerate}
    \item The limit on the right always exists: since $\b{x_n}$ is Cauchy, the sequence $\b{\n{x_n}}$ of reals is Cauchy, which implies it must converge.
    \item If $[x_n] = [y_n]$, then $\a{\n{x_n}-\n{y_n}} \le \n{x_n-y_n} \xrightarrow[n \to \infty]{} 0.$ Therefore, $\n{[x_n]} = \n{[y_n]}$.
  \end{enumerate}
  $X$ is embedded into $\wh X$ by mapping $x \in X$ into the class of the constant sequence at $x$.
  It is easy to see that this map preserves norms.
\end{proof}

\begin{theorem}
  Let $\wh X_2$ be another completion of $X$.
  Then exists a bijection $f \col \wh X \to \wh X_2$ which is linear, preserves norms,
  and maps the embedded $X$ into the embedded $X$.   
\end{theorem}

These two theorems endow us with the right to never consider pathological incomplete spaces.

\begin{proof}
  Map $[x_n] \in \wh X$ into $\lim x_n \in \wh X_2$.
\end{proof}

\begin{exercise}
  The space $C\q{a, b}$ with the norm
  $\n{f}_2 = \int \a{f}$ is not complete.
  Define
  $$
  f_n(x) = \begin{cases}
    0, & x \in [a, c], \\
    n(x-c), & x \in \q{c, c+ 1/n}, \\
    1, & x \in \q{c+1/n, b}.
  \end{cases}
  $$
  Then $f_n$ is a Cauchy sequence which does not have a limit in $C\q{a, b}$.
\end{exercise}

\begin{proof}
  It is easy to see that the limit of $\b{f_n}$ is $[x \ge c]|_{[a, b]}$, so it does not have a continuous limit. Nevertheless, it is Cauchy, since
  \begin{align*}
    \n{f_n - f_m}
    = \frac{1}{2} \a{\frac{1}{n} - \frac{1}{m}}
    \xrightarrow[\min\b{m,n}\to \infty]{} 0.
  \end{align*}
\end{proof}

\section{Equivalent norms}

\begin{definition}
  We say norms are \emph{equivalent}, iff the metrics they generate are Lipschitz equivalent.
\end{definition}

\begin{exercise}
  Norms are equivalent iff they generate the same topology.
\end{exercise}

\begin{exercise}
  In infinite-dimensional spaces, there are norms which are not equivalent.
\end{exercise}

\begin{proof}
  For example, consider $X = C\q{0, 1}$, $L^1$-norm and the sup-norm on it.
  It is true that $L^1$-norm does not surpass the sup-norm, but there is no constant for the opposite inequality: we can think of a function with an arbitrarily large sup-norm, but constant integral.
\end{proof}

\begin{theorem}
  If $X$ is finite-dimensional, then every two norms on $X$ are equivalent.
\end{theorem}

\begin{proof}
  Suppose $\dim X = n$, and $e_1, \dots, e_n$ is a basis.
  Let $x = a_1 e_1 + \dots + a_n e_n$.
  Let $\n{\square}$ be a norm on $X$.
  Define a new \emph{norm} as
  $$ \a{x} = \sqrt{a_1^2 + \dots + a_n^2}. $$
  Then
  \begin{align*}
    \n{x}
    & \le \sum_{i = 1}^n \a{a_i} \n{e_i} \\
    & \le M \sum_{i=1}^n \a{a_i} \\
    & \le M \p{\sum_{i=1}^n \a{a_i}}^\frac{1}{2} \\
    & = M \a{x}.
  \end{align*}
  
  The function $x \mapsto \n{x}$ is continuous in the norm $\a{\square}$.
  Let $\a{x_k - x} \to 0$.
  Then
  \begin{align*}
    \a{\n{x_k}-\n{x}}
    & \le \n{x_k-x} \\
    & \le M \sqrt{n} \a{x_k-x} \\
    & \to 0.
  \end{align*}
  
  Consider the set $S = \b{x \in E \mid \a{x} = 1}$.
  $S$ is compact in $\a{\square}$.
  $\phi|_S$ is continuous and nonzero.
  Then $\phi > \gd$ for some $\gd > 0$, so
  $$ \n{\frac{x}{\a{x}}} \ge \gd \iff \n{x} \ge \gd \a{x}. $$ 
\end{proof}

\begin{corollary}
  Every finite-dimensional normed vector space is complete.
\end{corollary}

\begin{proof}
  Every Euclidean space is complete.
\end{proof}

\begin{corollary}
  \label{finite-dimensional implies closed}
  A finite-dimensional subspace of a normed space is closed.
\end{corollary}

\begin{proof}
  It is complete, and every convergent sequence is Cauchy sequence.
\end{proof}

\begin{definition}
  The set $M \ss X$ is \emph{bounded}, iff
  $$ \sup_{m \in M} \n{m} < +\infty. $$
\end{definition}

\begin{lemma}[on an almost-perpendicular]
  Let $E$ be a normed space, and $F < E$ its closed proper subspace.
  Then for every $\eps > 0$ exists a vector $x \in E$ such that $\n{x} = 1$ and $\dist(x, F) > 1-\eps$.
\end{lemma}

\begin{proof}
  Let $y \in E \sm F$.
  Then $d = \on{dist}\p{y, F} > 0,$ since $F$ is closed.
  Let $\gd > 0$.
  By definition of infimum, there exists $a \in F$ such that
  $$ d \le \n{y - a} \le d + \gd. $$
  Put $y_2 = y-a$.
  Since $a \in F$, $\on{dist}\p{y_2, F} = \dist(y, F) = d$.
  Define
  $$ x = \frac{y_2}{d+\gd} = \frac{y-a}{d+\gd}. $$
  Then $\n{x} \le 1$, but
  $$ \on{dist}\p{x, F} = \dist\p{\frac{y}{d+\gd}, F} \ge \frac{\dist(y, F)}{d+\gd} = \frac{d}{d+\gd}. $$
  Since the $\gd$ is arbitrary, and by increasing the norm of $x$ we do not get closer to $F$, we get the desired.
\end{proof}

\begin{theorem}
  Let $X$ be a normed space. Equivalent are:
  \begin{enumerate}
    \item $X$ is finite-dimensional.
    \item Every bounded subset of $X$ is relatively compact.
  \end{enumerate}  
\end{theorem}

\begin{proof}[Proof of $1 \implies 2$.]
  From corollary on page \pageref{finite-dimensional implies closed}.
\end{proof}

\begin{proof}[Proof of $2 \implies 1$.]
  Suppose $X$ is not finite-dimensional.
  We assert that there exists a bounded sequence that does not have a convergent subsequence (so in no way the closure of a bounded subset that contains this sequence can be compact).
  We show this by induction: suppose $x_1, \dots, x_n$ are already built.
  By the almost-perpendicular lemma there exists $x_{n+1}$ such that
  $\n{x_{n+1}} = 1$ and
  $$ \dist\p{x_{n+1}, \Span\b{x_1, \dots, x_n}} > {1}/{2} $$
  (since $X$ is not finite-dimensional, the span here is a proper subspace of $X$).
  Continuing to infinity, we get a sequence $\b{x_n}$. It is bounded (all its members are on the unit sphere). Nevertheless, no subsequence of it is Cauchy by construction.
\end{proof}

\section{Linear operators}

\begin{definition}
  A linear operator $T \col X \to Y$ is \emph{bounded}, iff $T(B)$ is bounded, where $B$ is the unit ball in $X$.
\end{definition}

\begin{lemma}
  Let $X$ and $Y$ be normed vector spaces, and $T \col X \to Y$ a linear operator.
  The following are equivalent:
  \begin{enumerate}
    \item $T$ is continuous.
    \item $T$ is continuous at 0.
    \item $T$ is bounded.
  \end{enumerate}
\end{lemma}

\begin{proof}[Proof of $1 \Leftrightarrow 2$]
  Let $x \in X$. $T$ is continuous at $x$ iff for every convergent $x_n \to x$ the sequence $Tx_n$ also converges (to $Tx$).
  Now observe that
  \begin{align*}
    \n{Tx_n-Tx}
    &= \n{T\p{x_n-x}}.
  \end{align*}
\end{proof}

\begin{proof}[Proof of $2 \Rightarrow 3$]
  Consider $$ D = \b{y \in Y \mid \n{y} \le 1}. $$
  There is $\gd > 0$ such that, if $\n{x} \le \gd$, then $Tx \in D$.
  Let $z \in D$.
  Since
  $$\n{T{\gd z}} \le 1,$$ we have
  $$ \n{Tz} \le {1}/{\gd}. $$
\end{proof}

\begin{proof}[Proof of $3 \Rightarrow 1$]
  Let $\n{x} \le 1$. Then $\n{Tx} \le M$, so $\n{x} < \eps$ implies $\n{Tx} \le M \eps$.
\end{proof}

\subsection{Operator norm}

\begin{definition}
  Let $T$ is a continuous operator. Its \emph{norm} is
  $$ \n{T} = \sup_{x \ne 0} \frac{\n{Tx}}{\n{x}}. $$
\end{definition}

\begin{exercise}
  $\n{T}$ is indeed a norm.
\end{exercise}

\begin{lemma}
  An operator is bounded iff it has finite norm.
\end{lemma}

\begin{proof}
  Obvious.
\end{proof}

\begin{lemma}
  A linear combination of continuous operators is continuous.
\end{lemma}

\begin{proof}
  \begin{align*}
    \n{(kA+B)} \le \a{k}\n{A}+\n{B}.
  \end{align*}
\end{proof}

\subsection{The space of bounded operators is complete}

\begin{definition}
  $\mc{B}\p{X, Y}$ is the set of bounded linear operators $X \to Y$ with the obvious structure of a vector space and the standard operator norm as the norm.
\end{definition}

\begin{theorem}
  Let $Y$ be complete.
  Then $\mc{B}\p{X, Y}$ is complete.  
\end{theorem}

\renewcommand{\B}[2]{\mc{B}\p{{#1}, {#2}}}

{\footnotesize I have seen three quite concise proofs of this theorem and understood neither. Here is a long one, my own.}

\begin{proof}
  Let $\b{T_n}$ be a Cauchy sequence.
  Fix $x \in X$.
  $\b{T_n x}$ is a Cauchy sequence in $Y$:
  \begin{align*}
    \n{T_nx - T_mx}
    &\le \n{T_n-T_m}\n{x} \\
    &\le \eps \n{x}.
  \end{align*}
  By completeness of $Y$, there is a limit $t \xleftarrow[n \to \infty]{} T_n x$.
  
  The map $x \mapsto t$ we have just build is a linear operator. Call it $T$.
  $T_n$ is a Cauchy sequence, so $\n{T_n} \le M$ for some $M$.
  Then $T$ itself is bounded:
  \begin{align*}
    \n{Tx}
    &\le \n{T_nx}+ \eps\\
    &\le M \n{x} + \eps.
  \end{align*}
  
  We assert that $\n{T_nx-Tx} \to 0$.
  Suppose otherwise:
  \begin{align*}
    \ex \eps > 0\ \fa n_0 \in \N\ \ex n > n_0\ \ex x \in B \col \n{T_n x - Tx} > \eps.
  \end{align*}
  Since $\b{T_n}$ is Cauchy,
  \begin{align*}
    \fa \gd > 0\ \ex n_1 \in \N\ \fa k, l > n_1\ \fa x \in B \col \n{T_kx-T_lx} \le \gd.
  \end{align*}
  Fix this $\gd > 0$ and take the corresponding $n_1$.
  From the first line with quantifiers, there exist $n > n_1$ and $x \in B$ such that
  $$ \n{T_n x - Tx} > \eps. $$
  From the second one, for any $m > n_1$ we get
  \begin{align*}
    \n{T_n x - Tx}
    &\le \n{T_nx - T_mx} + \n{T_mx-Tx} \\
    &\le \gd + \n{T_mx-Tx}.
  \end{align*}
  Since $Tx = \lim_{m \to \infty} T_mx$, there is $m_0$ such that $\n{T_mx-Tx} \le \gd$ for all $m > m_0$.
  Take $m_1 = \max\b{m_0, n_1}$. Then, for any $m > m_1$,
  \begin{align*}
    \n{T_n x - Tx}
    &\le \gd + \n{T_mx-Tx} \\
    &\le 2\gd.
  \end{align*}
  Now launch $n_1 \to \infty$ and, consequently, $\gd \to 0$. We get a contradiction:
  \begin{align*}
    \eps < \n{T_n x - Tx} \le 2\gd.
  \end{align*}
\end{proof}

\begin{remark}
  Our proof does not use the boundedness of operators in the space $\mathcal{B}(X, Y)$.
\end{remark}

\section{Functionals}

\begin{definition}
  A \emph{functional} is a linear operator $X \to K$.
\end{definition}

\begin{definition}
  The \emph{dual} $X^*$ of $X$ is the space $\mathcal{B}(X, K)$ of continuous functionals.
\end{definition}

\begin{corollary}
  $X^*$ is complete.
\end{corollary}

\begin{proof}
  Since $K$ is complete in either case.
\end{proof}

\section{Strong convergence}

\begin{definition}
  A sequence of operators $T_n \to T$ converges \emph{strongly} or \emph{point-wise}, iff $T_n x \to Tx$ for all $x \in X$. 
\end{definition}

\begin{proposition}
  If $\n{T_n - T} \to 0$, then $T_n \to T$ strongly.
\end{proposition}

\begin{proof}
  Since
  \begin{align*}
    \a{T_nx - Tx}
    &\le \n{T_n-T}\a{x}.
  \end{align*}
\end{proof}

\begin{remark}
  The converse is not true.  
\end{remark}

\begin{definition}
  $l^p = L^p(\N, \#)$ is the space of real-valued sequences which converge in the $L^p$ norm (with respect to the cardinality measure). 
\end{definition}

\begin{proof}[Proof of the remark]
  Consider the operators
  \begin{align*}
    s_k (c_0, c_1, \dots) = (c_k, c_{k+1}, \dots)
  \end{align*}
  on $l^p$. $\n{s_k} = 1$, since there are sequences with a single unit and other elements zero and applying $s_k$ does not lessen the sequence norm anyway.
  Nevertheless, $s_k \to 0$ pointwise (strongly), since all the sequences in $l^p$ converge.
\end{proof}

\section{More on Banach spaces}

\begin{theorem}
  Let $Y$ be complete.
  Let $\b{T_n} \ss \mc{B}(X, Y)$ be operators with $\sup \n{T_n} < +\infty$, and let $E \ss X$ be dense.
  Suppose that, for every $e \in E$, the sequence $\b{T_ne} \ss Y$ converges.
  Then exists $T \in \mc{B}(X, Y)$ such that $T_n \to T$ pointwise.
\end{theorem}

\begin{remark}
  If $\b{T_n}$ converges strongly, then $\sup \n{T_n} < +\infty$.
  This is a harder fact.  
\end{remark}

\begin{proof}
  Define
  $$ Te := \lim_{n \to \infty} T_n e. $$
  Let $x\in x$. Take $\b{e_n}$ such that $\n{e_n-x} \xrightarrow[n\to\infty]{} 0$.
  We assert $T$ can be continued to a bounded operator $X \to Y$.
  
  $\b{Te_n}$ is Cauchy.
  \begin{align*}
    \n{Te_k-Te_k}
    &\le \n{Te_j-T_ne_j} + \n{T_ne_j-T_ne_k}+\n{Te_k-T_ne_k} \\
    &< 2\eps + \n{T_ne_j-T_ne_k} \\
    &\le 2\eps + \n{T_n}\n{e_j-e_k} \\
    &\xrightarrow[\substack{n\to \infty\\ \min\b{j,k}\to\infty}]{} 0.
  \end{align*}
\end{proof}

\begin{theorem}
  Let $X$ and $Y$ be normed spaces, and $Y$ complete.
  Let $F$ be dense in $X$.
  Let $T \col F \to Y$ be a continuous linear operator.
  Then exists unique $T_\sharp  \in \mc{B}(X, Y)$ such that $\n{T_\sharp} = \n{T}$ and $T_\sharp|_F = T$. 
\end{theorem}

\begin{proof}
  Let $x \in X$, $f_n \in F$, $\n{f_n-x} \to 0$.
  Then
  \begin{align*}
    \n{Tf_n - Tf_m}
    &\le \n{T} \n{f_n-f_m} \\
    &\xrightarrow[\min\b{m,n} \to \infty] 0.
  \end{align*}
  Hence $\b{Tf_n}$ is a Cauchy sequence and, as such, converges to some $y$.
  Define
  $$ T_\sharp x = y. $$
  This definition does not depend on the choice of the sequence.
\end{proof}

\section{Three fundamental principles of functional analysis}

\begin{enumerate}
  \item Hanh-Banach theorem.
  \item Closed graph and open map theorems.
  \item The principle of uniform boundedness.
\end{enumerate}


\section{Hanh-Banach theorem}

\begin{definition}
  Let $X$ be a vector space over $K$, $p \col X \to \R$.
  It is said that $p$ is a \emph{seminorm}, iff
  \begin{enumerate}
    \item $p(x) \ge 0$.
    \item $\a{k}p(x) = p(kx)$ for all $k \in K$.
    \item $p(x+y) \le p(x)+p(x)$.
  \end{enumerate}
\end{definition}

\begin{example}
  Consider $X = C\p{\R}$ and $p(f) = \int_0^1 \a{f}$. Then $p$ is a seminorm, but not a norm, since there is a nonzero function, which is zero on $[0, 1]$.
\end{example}

\begin{definition}
  Let $X$ be a $K$-vector space. A $p \col X \to \R$ is a \emph{sublinear functional}, iff
  \begin{enumerate}
    \item $p(x+y) \le p(x)+p(y)$.
    \item $p(kx) = \a{k}p(x)$ for all $k \in K$.
  \end{enumerate}
\end{definition}

\begin{theorem}[Hanh, Banach]
  Let $X$ be a real vector space, $p \col X \to \R$ a sublinear functional.
  Let $Y \le X$ and $f \col Y \to \R$ a linear functional such that $f(y) \le p(y)$ for all $y \in Y$.
  Then exists $F \col X \to \R$ such that $F|_Y = f$ and $F(x) \le p(x)$ for all $x \in X$.
\end{theorem}

\begin{lemma}[Hanh-Banach in codimesion 1]
  Ler $x_0 \in X \sm Y$.
  Let $Y_\sharp = Y + \Span\b{y_0}$.
  Then $f$ can be continued to a linear functional on $f_\sharp  \col Y_\sharp \to \R$ such that $f_\sharp \le p$ on $Y_\sharp$.
\end{lemma}

\begin{proof}
  Let $y \in Y_\sharp$, $y \in Y$, $\ga \in \R$.
  
  We assert a $c \in \R$ can be chosen in such a way that
  $$ f_\sharp(y+\ga y_0) := f(y) + \ga c $$
  satisfies
  \begin{equation}
    \label{j1}
    f_\sharp \le p \iff f(y)+\ga c \le p\p{y+\ga y_0}.
  \end{equation}
  
  If $\ga = 0$, the inequality is satisfied.
  
  Suppose $\ga > 0$.
  Divide \eqref{j1} through by $\ga$:
  $$ f\p{\frac{y}{\ga}} + c \le p\p{\frac{y}{\ga}+y_0}. $$
  This rewrites as
  $$ p\p{y_1+y_0} - f\p{y_1} \ge c, $$
  where $y_1= {y}/{\ga}$.
  
  Suppose $\ga < 0$. Dividing \eqref{j1} through by $-\ga$, we get
  $$ f\p{y_2}-p\p{y_2-y_0} \le c, $$
  where $y_2 = -y/\ga$.
  
  Now, if we show that
  $$ f\p{y_2}-p\p{y_2-y_0} \le p\p{y_1+y_0} - f\p{y_1}, $$
  we are done by the Cantor-Dedekind axiom. But that trivially follows from the triangle inequality for $p$ and linearity of $f$.
\end{proof}

\begin{proof}[Proof of the theorem of Hanh and Banach in the general case]
  Consider the set
  $$ A = \b{(M, f_M) \mid Y \le M \le X,\ f_M \col M \to \R \text{ is a linear functional},\ f_M \le p}. $$
  We tell that $(M, f_M) \le (N, f_N)$, iff $M \le N$ and $f_N|_M = f_M$.
  The union of any chain is its supremum; we have a maximal element $(L, F)$.
  We assert that $L = X$.
  
  Suppose otherwise, and let $y_0 \in X \sm L$.
  Define $L_\sharp = L + \Span\b{y_0}$.
  By the lemma, there exists $F_\sharp \col L_\sharp \to \R$ such that $F_\sharp \le p.$
  But then $(L_\sharp, F_\sharp) \ge (L, F)$. 
\end{proof}

\newcommand{\cB}[0]{\mc{B}}

\subsection{Useful corollaries}

\begin{corollary}
  Let $Y \le X$ and $f \in \mc{B}(Y, \R)$. Then there exists $F \col X \to \R$ such that $F|_Y = f$ and $\n{F} = \n{f}$.
\end{corollary}

\begin{proof}
  Let $p(x) = \n{f} \n{x}$. Then $f(y) \le p(y)$ for all $y \in Y$.
  Take $F$ as in the HB theorem.
  Then
  $$ F(x) \le \n{f} \n{x}. $$
  Likewise, taking the $-F$ for $-f$, we get
  $$ \n{F} \le \n{f}. $$
  The converse inequality is evident.
\end{proof}

\begin{corollary}
  \label{a point and a linear subspace can be separated}
  Let $Y \le X$, $x_0 \in X \sm Y$. Then exists $F \in X^*$ such that $\n{F} \le 1$, $F|_Y = 0$, and $F\p{x_0} = \dist\p{x_0, Y}$.
\end{corollary}

\begin{proof}
  Define $p(x) = \dist(x, Y)$.
  Since $Y$ is closed, $d := p(x_0) > 0$.
  Define
  $$ f(y+\ga x_0) := \ga d. $$
  Then $f \le p$, and exists $F \col X \to \R$ such that $F|_L = f|_L$ and $F \le p$.
  In particular, $F|_Y = 0$ (what we need) and $F(x_0) = d$.
  Observe that the same applies to $-f$, and so $\n{F(x)} \le p(x) \le \n{x}$.
  Hence $\n{F} \le 1$.
\end{proof}

\begin{corollary}
  Let $x_0 \in X$. Then exists $F \in X^*$ such that $\n{F(x_0)} = \n{x_0}$ and $\n{F} = 1$.
\end{corollary}

\begin{proof}
  The case $Y = \b{0}$ of the previous corollary.
\end{proof}

\begin{corollary}
  Let $X$ be a normed space, $M \le X$.
  Then
  $$ \Cl M = \bigcap \b{ \ker f \mid f \in X^*,\ M \ss \ker f }. $$  
\end{corollary}

\begin{proof}
  Let $N$ denote the intersection.
  Since $N$ is closed as an intersection of closed sets, $\Cl M \ss N$.
  We now show that $\ol{\Cl M} \ss \ol{N}$.
  Let $x_0 \not\in \Cl M$.
  By a corollary from page \pageref{a point and a linear subspace can be separated}, there exists a functional $f \on X^*$ such that $\n{f} \le 1$, $M \ss \ker f$ and $f(x_0) = \dist(x_0, M)$.
  $f$ participates in the intersection on the right, but its kernel does not contain $x_0$.
  Therefore, $x_0 \not\in N$, as asserted.
\end{proof}

\begin{corollary}
  Let $X$ be a normed space, $M \le X$.
  The following conditions are equivalent:
  \begin{enumerate}
    \item $M$ is dense.
    \item For every functional $f \in X^*$, if $M \ss \ker f$, then $f = 0$.
  \end{enumerate}
\end{corollary}

\begin{proof}
  This is a reformulation of the previous theorem.
\end{proof}

\section{Banach limits}

\begin{definition}
  Let $c \le l^\infty$ be the space of convergent subsequences.
  A map $L \col l^\infty \to \R$ is a \emph{Banach limit}, iff
  \begin{enumerate}
    \item It is a continuous linear functional.
    \item For all $x = \b{x_n} \in c$, $$ Lx = \lim x_n. $$
    \item If $x \ge 0$, then $Lx \ge 0$.
    \item $L\b{x_{n+1}} = L\b{x_n}$.
  \end{enumerate}
\end{definition}

\begin{theorem}
  $L$ exists.
\end{theorem}

\begin{proof}
  Put, for $x = \b{x_n}$,
  $$ f(x) = \lim x_n. $$
  $f \col c \to \R$ is a linear functional.
  Define a majoring sublinear functional (exercise) as
  $$ p(x) = \limsup \frac{1}{n} \sum_{i=1}^n x_i. $$
  Observe that $p|_c = f$, so $f \le p$.
  By HB, we have $L \col l^\infty \to \R$, which, luckily, satisfies the definition of Banach limit:
  \begin{enumerate}
    \item $L|_c = f$.
    \item If $x \le 0$, then $Lx \le p(x) \le 0$.
    \item Put $x'_i = x_{i+1}-x_i$. By linearity of $L$, it is sufficient to show that $L\b{x'_n} = 0$. And indeed,
      $$ p(x') = \lim \frac{1}{n} \sum_{i=1}^n \p{x_{n+1}-x_n} = 0. $$
  \end{enumerate}
\end{proof}

\subsection{The complex case}

\begin{definition}
  Let $c \le l^\infty(\C)$ be the space of convergent subsequences.
  A map $L \col l^\infty \to \C$ is a \emph{Banach limit}, iff
  \begin{enumerate}
    \item It is a continuous linear functional.
    \item For all $x = \b{x_n} \in c$, $$ Lx = \lim x_n. $$
    \item If $x \in \R$ and $x \ge 0$, then $Lx \ge 0$.
    \item $L\b{x_{n+1}} = L\b{x_n}$.
    \item $\n{L} = 1$.
  \end{enumerate}
\end{definition}

\begin{theorem}
  $L$ exists.  
\end{theorem}

\begin{proof}
  Let $L$ be a real Banach limit.
  For $a, b \in l^\infty(\C)$, define
  $$ L(a+ib) := La + iLb. $$
  All of the properties now follow trivially from those of real limits, except the final one.
  
  Simple functions are dense in $l^\infty$, so we may prove the statement for them and be happy after using continuity of $L$.
  Let $x = \sum \ga_k \chi_{E_k}$ for some partition $\bigsqcup_{k \in \N} E_k = \N$, and $\a{\ga_\s} \le 1$.
  Then $Lx = \sum \ga_k L\chi_{E_k}$, and
  \begin{align*}
    \a{Lx}
    &\le \sum_{k \in \N} L\p{\chi_{E_k}} \\
    &= L\p{\chi_{\sqcup E_k}} \\
    &\le 1,
  \end{align*}
  which was asserted.
\end{proof}

\section{Complex Hanh-Banach}

Let $K = \C$.

\begin{definition}
  Let $Z$ be a complex vector space.
  We denote by $Z^\star$ the space of bounded linear functionals $Z \to \C$.
\end{definition}

\begin{lemma}
  A functional $f \in Z^\star$ can be recovered from its real or imaginary part.
\end{lemma}

\begin{proof}
  Let $f = u +iv$. Since $f$ is linear over $\C$, we may write
  $$ if(x) = iu(x)-v(x) = f(ix) = u(ix)-iv(ix). $$
  Since $1$ and $i$ are a basis, we infer from here that
  $$ -v(x) = u(ix). $$
\end{proof}

\begin{theorem}[Hanh-Banach, the complex version]
  Let $X$ be a complex normed space, $Y \le X$.
  Let $p \col X \to \R$ be a seminorm.
  Let $f \in Y^\star$ be a linear functional such that $\a{f} \le p$.
  Then exists a functional $F \in X^\star$ such that $F|_Y = f$ and $\a{F} \le p$.
\end{theorem}

\begin{proof}
  Let $f \in Y^\star$ and $u = \Re f$.
  $u$ is a real linear functional on $Y$.
  By the real HB, exists a functional $U \in X^*$ such that $U|_Y = u$ and $\a{U(x)} \le p(x)$ for all $x \in X$.
  Define
  $$ F(x) := U(x) - iU(ix). $$
  This is a complex functional on $X$, and $F|_Y = f$.
  We assert that $\a{F} \le p$.
  Let $r \in C$ be such that
  $$ \a{F(x)} = r F(x) $$
  and $\a{r} = 1$.
  Then $$ \a{F(x)} = F(rx) = U(rx) \le p(rx) = \a{r} p(x) = p(x), $$
  what was to be shown.
\end{proof}


\section{Quotient spaces}

\begin{definition}
  Let $X$ be a normed space, $M \le X$.
  On $X/M$ we can introduce a norm:
  $$ \n{[x]} := \dist(x, M). $$
  In this section, we preserve this naming, using, furthermore, $[\s]$ or $q \col X \to X/M$ for the canonical projection. (And of course, we use choice throughout.)
\end{definition}

\begin{lemma}
  This is indeed a norm.
\end{lemma}

\begin{proof}
  ~\begin{enumerate}
    \item \emph{Homogeneity.} $x = x'+m$ for some $m \in M$, $x' \in \ol M$.
    $m$ is the closest to $x$ point of $M$: for any other $m_1$ we have
    $$ \n{x-m_1} = \n{x'+m-m_1} \ge \n{m-m_1}+\n{x'}. $$ 
    Then
    $$ \n{[kx]} = \n{kx-m} = \a{k} \n{x'} = \a{k} \n{[x]}. $$
    \item \emph{Triangle inequality}. Since
    $$ \n{x+y-m} \le \n{x-m/2} + \n{y-m/2}. $$
    \item \emph{Nonzero on nonzero vectors.} By closedness of $M$.
    \item \emph{Zero on the zero vector.} As $0 \in M$.
  \end{enumerate}
\end{proof}

\begin{lemma}
  $\n{[x]} \le \n{x}$ for all $x \in X$.
\end{lemma}

\begin{proof}
  $\dist(x, M) \le \n{x}$.
\end{proof}

\begin{lemma}
  If $X$ is complete, then $X/M$ is.
\end{lemma}

\begin{proof}
  Let $\b{[x_n]}$ be a Cauchy sequence.
  There exists a subsequence $\b{x_{n_k}}$ such that
  $$ \n{\q{x_{n_k}-x_{n_{k+1}}}} < 2^{-k}. $$
  Let $\b{y_k}$ be a sequence in $M$ such that
  $$ \n{x_{n_k}+y_k-x_{n_{k+1}}-y_{k+1}} < 2^{1-k} $$
  (it exists by the previous inequality and the definition of distance).
  Then $\b{x_{n_k}-y_k}$ is a Cauchy sequence (\textbf{exercise}).
  Since $X$ is complete, it converges to some $x_*$.
  Then
  $$ \q{x_{n_k}} \to [x_0] $$
  by the previous lemma.
\end{proof}

\begin{lemma}
  Let $W \ss X/M$.
  $q^{-1}(W)$ is open iff $W$ is open.
\end{lemma}

Hence the norm topology we have introduced equals the usual quotient topology.

\begin{proof}[$\Leftarrow$]
  Let $[x] \in W \ss X/M$.
  We want to find a neighbourhood $U \ni x$ such that $[U] \ss W$.
  
  Since $W$ is open,
  there exists $\eps > 0$ such that, if $\n{[y]-[x]}$, then $qy \in W$.
  
  That is, if $\dist(y, x) < \eps$, then $y \in q^{-1}(U)$; as needed.
\end{proof}

\begin{proof}[$\Rightarrow$]
  Suppose that $q^{-1}(W)$ is open.
  Let $[x] \in W$.
  We want to find a ball around $[x]$ that would lie in $W$.
  There is one around $x$ that lies in $q^{-1}(W)$: for all $y$ with $\n{y-x}<$ some $\eps > 0$ we have $y \in q^{-1}(W)$.
  This implies, in particular, that
  $$ \n{[y]-[x]} \le \n{y-x} < \eps. $$
\end{proof}

\begin{lemma}
  Let $U \ss X$.
  Then $qU$ is open.
\end{lemma}

That is, $q$ is open.

\begin{remark}
  The converse is not necessarily true.  
\end{remark}

\begin{proof}
  Sufficient to show that
  $ \text{$q^{-1}\p{qU}$ is open}. $
  In fact,
  $$ q^{-1}(qU) = \bigcup_{y \in M}(U+y). $$
\end{proof}

\subsection{Annulator}

\begin{definition}
  Let $M \le X$, as before.
  Its \emph{annulator} is the set
  $$ M^\perp = \b{f \in X^* \mid M \ss \ker f}. $$
\end{definition}

\begin{theorem}
  There exists a linear isometry
  $$ \gr \col X^*/M^\perp \longrightarrow M^*, $$
  defined by
  $$ \gr \col [f] \longmapsto f|_M. $$
\end{theorem}

\begin{proof}
  Let $[f] \in X^*/M^\perp$.
  
  The $\gr$ defined is indeed a bijective function, since
  for every $g \in M$ we have
  \begin{align*}
    f \sim g
    &\iff f-g \sim 0 \\
    &\iff f-g \in M^\perp \\
    &\iff \p{f-g}|_M =0 \\
    &\iff f|_M = g|_M.
  \end{align*}

  This function does map into $M^*$:
  $f|_M \in M^*$ (a restriction of a linear function onto a linear subspace is linear; a restriction of a continuous function is continuous).
  
  We show that it is an isometry.
%  One way:
%  \begin{align*}
%    \n{[f]}
%    &= \dist\p{f, M^\perp} \\
%    &= \inf \b{ \n{f-g} \mid g|_M = 0 } \\
%    &\ge \inf \b{ \n{\p{f-g}|_M} \mid g|_M = 0 } \\
%    &= \inf \b{ \n{f|_M} \mid g|_M = 0 } \\
%    &= \n{f|_M}.
%  \end{align*}
  Let $\gl \in M^*$. By HB there exists $\Lambda \col X \to K$ such that $\Lambda|_M = \gl$ and $\n{\Lambda} = \n{\gl}$. But $\Lambda$ gets mapped into $\gl$ by $\rho$, and $\rho$ is bijective.
\end{proof}

\newcommand{\Gl}[0]{\Lambda}

\begin{theorem}
  There exists a linear isometry
  $$ \p{X / M }^* \longleftrightarrow M^\perp $$
  defined by
  $$ \rho \col f \longmapsto f \circ Q. $$
\end{theorem}

\begin{proof}
  Let $\gl \in \p{X/M}^*$, and let $\Lambda := \rho(\gl).$ 
  Since the norm of the composition does not surpass the product of norms, we have
  $$ \n{\Lambda} \le \n{\gl}. $$
  Also, note that
  $ \Lambda_|M  = 0. $
  
  Let $\Lambda \in M^\perp$.
  Define $\gl[x] := \Lambda x$. 
  Then $\n{\gl} = \n{\Gl}$: (1) obviously, $\n{\gl} \le \n{\Gl}$; (2) the other inequality has been shown.
\end{proof}

\section{Separability of duals}

\begin{theorem}
  Let $X$ be normed and $X^*$ separable.
  Then $X$ is separable.
\end{theorem}

\begin{example}
  The converse is false.
  Let $X = l^1$. Then $\p{l^1}^* \cong l^\infty$. But $l^1$ is separable, while $l^\infty$ is not.
\end{example}

\begin{proof}
  Let $\b{f_j}$ be a dense sequence in the unit sphere of $X^*$.
  Such exists by hypothesis.
  
  By definition of norm in the dual, there is a sequence $\b{x_j} \in X$ with $\n{x_j} = 1$ and $\a{f_j(x_j)} \ge 1 - \eps_j$ for all $j$.
  
  Let $M = \Cl \Span \b{x_j}$. By definition, this is a separable space.
  
  Let $y\not\in M$. By one of corollaries of HB there exists $f \in X^*$ such that $f|_M = 0$, $f(y) \ne 0$, $\n{f} = 1$.
  
  Since $\b{f_j}$ is dense, there exists a convergent subsequence $g_k$ of $\b{f_j}$: $g_k \to f$ .
  Assume that $\a{g_k(x_k)} \ge 1/2$. 
  Then
  \begin{align*}
    \a{f(g_k)}
    &= \a{f(x_k) - g_k(x_k) + g_k(x_k)} \\
    &\ge \a{g_k(x_k)} - \a{f(x_k)-g_k(x_k)} \\
    &\ge 1/2 - o(1).
  \end{align*}
  This is in contradiction to $f|_M = 0$.
\end{proof}

\section{Open map and closed graph theorems}

\begin{theorem}[on an open map]
  Let $X$ and $Y$ be Banach and $A \in \mc{B}(X, Y)$.
  Then $A$ is open.
\end{theorem}

The proof spans several lemmas.

Let $B_r \ss X$ be the ball of radius $r > 0$ with centre at $0$.

\begin{lemma}
  $0 \in \Int \Cl A(B_1)$.
\end{lemma}

\begin{proof}
  Observe that
  $$ Y = \bigcup_{n \ge 1} \Cl A(B_n). $$
  By Baire's theorem, one of the sets in the union has non-empty interior.
  But as they are all homothetic, this is true for any of them.
  Suppose $y_0$ lies in $V := \Int \Cl A(B_1)$ together with a ball $B$ around it of radius $\gd$.
  
  Let $\b{x_n}$ be such that $\n{x_n} < 1$ and $Ax_n \to y_0$ (such exists, since $y_0$ is in the closure).
  
  For sufficiently small $y$, $y_0 + y \in \Cl A(B_1)$.
  
  There exists $\b{z_n}$ such that $\n{z_n} < 1$ and $Az_n \to y_0 + y$.
  $A(z_n-x_n) \to y$.
  Since $z_n, x_n \in B_1$, $z_n-x_n \in B_2$.
  The last two sentences say exactly that
  $$ y \in \Cl A(B_2). $$
  This holds for sufficiently small $y$, which is equivalent to saying that
  $$ 0 \in \Int \Cl A(B_1). $$
\end{proof}

\begin{lemma}
  $\Cl A(B_1) \ss A(B_2)$.
\end{lemma}

\begin{proof}
   Let $y_1 \in \Cl A(B_1)$.
   $0 \in \Int \Cl A(B_{1/2})$.
   Since in every neighbourhood of $y_1$ there are points from $A(B_1)$,
   $$ \p{y_1-\Cl A(B_{1/2})} \cap A(B_1) \ne \naught. $$
   That is, exists $x_1 \in B_1$ such that
   $$ Ax_1 \in y_1 - \Cl A(B_{1/2}). $$
   For some $y_2 \in \Cl A(B_{1/2})$,
   $$ Ax_1 = y_1-y_2. $$
   Likewise, 
   $$ \p{y_2-\Cl A(B_{1/4})} \cap A(B_{1/2}) \ne \naught. $$
   Continuing to infinity, we get a sequence $\b{x_n}$ such that
   $$ \n{x_n} \le 2/2^n, $$
  and a sequence $\b{y_n}$ such that $y_n \in \Cl A\p{B_{2/2^n}}$ and
   $$ Ax_n = y_n-y_{n+1}. $$
   Let $x := \sum x_n$ --- by completeness of $X$, this vector indeed exists and has norm $< 2$.
   By continuity of $A$, $y_1 = Ax \in A(B_2)$.
\end{proof}

\begin{proof}[Proof for the open graph theorem]
  Summing the two lemmas, we get
  $$ 0 \in \Int \Cl A(B_1) \ss \Int A(B_1). $$
  This finishes the proof, since
  $$ \n{Ax-Ax_0} < \eps \iff \n{A(x-x_0)} < \eps. $$
\end{proof}

\subsection{Inverse function theorem}

\begin{theorem}[inverse function theorem]
  Let $X$ and $Y$ be Banach and $A \in \mc{B}(X, Y)$ a bijection.
  Then $A^{-1}$ is linear and continuous.
\end{theorem}

\begin{proof}
  $A^{-1}$ is linear anyway.
  Continuity is a direct corollary of the open map theorem.
\end{proof}

\section{Closed graph theorem}

\begin{definition}
  If $X$ and $Y$ are normed spaces, their coproduct has the following structure of a normed space:
  $$ \n{x \oplus y} := \n{x} + \n{y}. $$
\end{definition}

\begin{lemma}
  Let $A \col X \to Y$ be a linear operator. Its graph is closed iff for any sequence $x_n \to 0$ such that the limit $\lim Ax_n$ exists, $Ax_n \to 0$.
\end{lemma}

\begin{remark}
  The difference with continuity is that we do not expect that $\b{Ax_n}$ will automatically converge.
  The closed graph theorem tells it will in complete spaces.
\end{remark}

\begin{proof}[$\Rightarrow$.]
  As the graph is closed, there is some $x$ so that $(x_n, Ax_n) \to (x, Ax)$.
  $$ \n{x_n-x}+\n{A(x_n-x)} \to 0. $$
  Then
  $$ \n{x_n-x}, \n{A(x_n-x)} \to 0. $$
  Since the sequence limit is unique, $x = 0$.
  Then $\n{Ax_n} \to 0$.
\end{proof}

\begin{proof}[$\Leftarrow$.]
  Let $(x_n, Ax_n) \to (x, y)$. We assert that $y = Ax$.
  Indeed,
  $$ \n{x_n-x}+\n{Ax_n-y} \to 0 $$
  implies that $x_n-x \to 0$.
  But then $A(x_n-x) \to 0$ by hypothesis, and so
  $$ \n{Ax_n -Ax} \to 0. $$
  By uniqueness of limits, $y = Ax$.
\end{proof}

\begin{theorem}[the closed graph theorem]
  Let $X$ and $Y$ be complete, $A \col X \to Y$ linear.
  If the graph of $A$ is closed, then $A$ is continuous.
\end{theorem}

The converse is obvious.

\begin{proof}
  $X \otimes Y$ is complete.
  Let $G$ be the graph of $A$.
  Let $P \col G \to X$ be defined as 
  $$ P(x, Ax) := x. $$
  $P$ is a bijection.
  The inverse function theorem says that $P^{-1}$ is continuous.
  Let $Q \colon G \to X$ be defined as
  $$ Q(x, Ax) := Ax. $$
  Then $Q$ is continuous.
  Then $A = Q P^{-1}$ is continuous.
\end{proof}

\subsection{Why is the closed graph theorem important}

\begin{lemma}
  Let $\phi \in L^p$.
  Define a linear $M \col L^p \to L^p$ as
  $ M f := \phi \cdot f$.
  Suppose that $\im M \ss L^p$.
  Then $M$ is bounded.
\end{lemma}

\begin{proof}
  We check that the graph of $M$ is closed.
  Let $f_n \to 0$ in $L^p$ and $\phi f_n \to g$.
  By Hölder's inequality, the sequence $\phi f_n$ also converges to 0.
\end{proof}

\subsection{Sobolev spaces}

\begin{definition}[Sobolev spaces]
  The normed space $W^n_p[a, b]$ consists of $f \in C^{n-1}[a,b]$ with the norm
  $$ \n{f}_{W^n_p} := \p{\sum_{k=0}^n \n{f^{(k)}}_{L^p}^p}^{\frac{1}{p}}. $$
\end{definition}

\begin{lemma}
  This is a complete space.
\end{lemma}

\begin{theorem}
  Let $M \col f \mapsto \phi \cdot f$ for some $\phi \in W^n_p$. Then $M$ is bounded.
\end{theorem}

\begin{proof}
  Let $f_n \to 0$ and $\phi f_n \to g$.
  Then $f_n \to 0$ pointwise. 
  Then $\phi f_n \to 0$ pointwise, which implies $g = 0$.
\end{proof}

\subsection{Complementary spaces}

\begin{definition}
  Let $X$ be Banach.
  Let $M, N \le X$ be closed spaces.
  The $M$ and $N$ are said to be \emph{algebraically complementary}, iff
  $$ M+N=X \qquad\text{and}\qquad M \cap N = \b{0}. $$
  They are \emph{topologically complementary}, iff, in addition,
  $$ \n{x \oplus y} \sim \n{x}+\n{y} $$
  for all $x \in M$ and $y \in N$ (that is, these norms are equivalent on $M \otimes N \cong X$.
\end{definition}

\begin{theorem}
  If $X$ is Banach and $M, N \le X$ are algebraically complementary, then $M$ and $N$ are topologically complementary.
\end{theorem}

\begin{proof}
  The map
  $$ x + y \mapsto x \otimes y $$
  is an isomorphism of $M+N$ and $M \otimes N$.
  The norms $\n{x}+\n{y}$ and $\n{x+y}$ are equivalent.
  The inverse map is continuous by the inverse function theorem; hence Lipschitz.
  Together with the triangle inequality, this gives the equivalence of norms.
\end{proof}

\section{Uniform boundedness principle}

\begin{theorem}[uniform boundedness principle]
  Let $X$ be Banach and $Y$ normed.
  Let $A \ss \mc{B}(X, Y)$.
  $A$ is bounded
  iff
  $$ \fa x \in X \col  \sup_{a \in A} \n{ax} < \infty. $$
\end{theorem}

\begin{proof}
  If $A$ is bounded, this is trivially true. We prove the converse.
  
  Let
  $$ Q_n := \b{x \in X \mid \fa a \in A \col \n{ax} \le n }. $$
  $Q_n$ is closed as an intersection of closed spaces (all $a$ are continuous).
  By Baire's theorem, $\bigcap Q_n$ has non-empty interior.
  Let $x_0 \in \bigcap Q_n$ be such that
  $$ \n{a(x_0+x)} \le 1 $$
  whenever $\n{x} < r$ for all $a \in A$.
  Then
  $$ \n{ax} \le 1 + \n{ax_0} $$
  for all $a \in A$.
  This implies
  $$ \sup_{\n{x} < n} \n{ax}  < +\infty. $$
\end{proof}

\begin{corollary}
  Let $X$ be normed, $A \ss X$.
  $A$ is bounded iff, for every $f \in X^*$, $f(A)$ is bounded.
\end{corollary}

\begin{proof}
  The dual is complete.
  Put $X \from X^*$ and $Y \from X$. 
\end{proof}

\begin{theorem}
  Let $X$ be Banach, $Y$ normed, and $A \ss \mc{B}(X, Y)$.
    $A$ is bounded iff
    $$ \fa x \in X\ \fa g \in Y^* \col \sup_{a \in A} \a{gax} < \infty. $$  
\end{theorem}

\begin{proof}
  $A$ is bounded iff $gA$ is bounded for every $g$ iff
  $$ \fa g \in X^*\ \fa x \in X \col \sup_{ga \in gA} \a{gax} < \infty $$
  iff
  $$ \fa g \in X^*\ \fa x \in X \col \sup_{a \in A} \a{gax} < \infty. $$
\end{proof}

\section{Banach-Steinhaus theorem}

\begin{theorem}
  Let $X, Y$ be Banach spaces.
  Let $A_n \in \mc{B}(X, Y)$.
  Suppose $A_n x$ converges in $Y$ for all $x \in X$.
  Then
  $$ \sup_n \n{A_n} < \infty, $$
  and exists $A \in \mc{B}(X, Y)$ such that
  $$ \n{A_n x - Ax} \to 0 $$
  for all $x \in X$.
\end{theorem}

\begin{proof}
  Put $Ax := \lim\p{A_n x}$.
  If $A_n x$ converges for all $x$, then $A_n x$ is bounded in $Y$ for all $x$.
  Then $\b{A_n}$ is bounded, and so
  $$ \n{Ax} \le \sup_{n} \n{A_n} \n{x}. $$ 
\end{proof}

\section{Inner products}

\begin{definition}
  Let $H$ be a vector space over $K$.
  An \emph{inner product} on $H$ is a map $\i{\s,\s} \col H^2 \to K$ such that
  \begin{enumerate}
    \item $\i{\ga x + \gb, y} = \ga \i{x, y} + \gb$ for all $x, y \in H$ and $\ga, \gb \in K$.
    \item $\i{x, y} = \ol{\i{y, x}}$ for all $x, y \in H$.
    \item $\i{x, x} \in \R$ and $\i{x, x} \ge 0$.
    \item $\i{x, x} = 0$ implies $x = 0$.
  \end{enumerate}
\end{definition}

\begin{lemma}[CBS]
  $$ \a{\i{x,y}} \le \n{x} \n{y}. $$
\end{lemma}

\begin{proof}
  Let $\ga \in K$.
  \begin{align*}
    0
    &\le \i{x+\ga y, x+\ga y} \\
    &= \n{x}^2 + 2 \Re\p{\ga \i{y, x}} + \a{\ga}^2 \n{y}^2.
  \end{align*}
  $\ga =: re^{i\phi}$. Choose $\phi$ in such a way that $\ga \i{y, x} \in \R$.
  The resulting degree 2 polynomial in $r$ has at most one root.
\end{proof}

\begin{definition}
  $$ \n{x} := \sqrt{\i{x, x}}. $$
\end{definition}

\begin{lemma}
  This is a norm.
\end{lemma}

\begin{proof}
  We check triangle inequality.
  \begin{align*}
    \n{x+y}^2
    &\le \p{\n{x}+\n{y}}^2 \\
    \iff \n{x}^2 + 2 \Re \i{x, y} + \n{y}^2 &\le \n{x}^2 + 2 \n{x}\n{y} + \n{y}^2.
  \end{align*}
  This follows from CBS.
\end{proof}

\begin{lemma}[parallelogram identity]
  $$ \n{x+y}^2 + \n{x-y}^2 = 2\p{\n{x}^2 + \n{y}^2}. $$
\end{lemma}

\begin{proof}
  Direct computation.
\end{proof}

\begin{lemma}
  Let $p \ne 2$ and $p \ge 1$.
  Then the $l^p$ norm is not induced by any inner product.
\end{lemma}

\begin{proof}
  $2^{1/p} + 2^{1/p} \ne 2 (1 + 1). $
\end{proof}

\begin{definition}
  We say $x$ and $y$ are \emph{orthogonal} and write $x \perp y$, iff $\i{x, y} = 0$.
\end{definition}

\begin{lemma}[Pythagoras]
  $x_1, \dots, x_n$ are pairwise orthogonal iff
  $$ \n{x_1+ \dots + x_n}^2 = \n{x_1}^2 + \dots + \n{x_n}^2. $$
\end{lemma}

\begin{proof}
  Direct computation.
\end{proof}

\begin{lemma}[polarisation identity]
  Let $U = \b{ \pm 1, \pm i}$ in case $K= \C$ and $U = \b{ \pm 1}$ in case $K = \R$. Then
  $$ \i{x, y} = \frac{1}{4} \sum_{\ga \in U} \ga \n{x + \ga y}^2. $$
\end{lemma}

\begin{proof}
  Direct computation.
\end{proof}

\begin{lemma}
  Let $f \col U \to B$ a linear operator.
  $\n{fu} = \n{u}$ for all $u \in U$ iff
  $\i{fx, fy} = \i{x, y}$.
\end{lemma}

\begin{proof}
  From the polarisation identity.
\end{proof}

\section{Hilbert spaces}

\begin{definition}
  Let $H$ be an inner product space.
  $H$ is \emph{Hilbert}, iff it is complete with respect to the norm that is induced by the inner product.
\end{definition}

\begin{example}
  Consider $C[a,b]$ with
  $$ \i{f,g} := \p{ \int \ol f \cdot g }^{1/2}. $$
  This space is not complete, so not Hilbert.
  But its completion is isometrically isomorphic to $L^2[a, b]$.
\end{example}

\section{Projections}

\begin{theorem}[existence and uniqueness of projections]
  Let $H$ be a Hilbert space, $C$ a closed convex set, $a \in H$.
  Then exists unique $x_0 \in C$ such that
  $$ \n{a-x_0} = d := \dist \p{a, C}. $$
\end{theorem}

\begin{proof}[Proof of existence]
  We may assume that $a = 0$.
  We want a vector of minimal norm in $C$.
  Exists a finite sequence $x_n \in C$ such that $\n{x_n} \to d$.
  From parallelogram identity,
  $$ \n{\frac{x_n-x_m}{2}}^2 + \n{\frac{x_n+x_m}{2}}^2 = \frac{\n{x_n}^2+\n{x_m}^2}{2}. $$
  The right part tends to $d^2$; the right addend on the left part is at least $d^2$. Then $x_n \to x_0$ for some $x_0 \in C$, since $H$ is complete.
\end{proof}

\begin{proof}[Proof of uniqueness]
  From the same identity we derive that the difference between two such vectors must be zero.
\end{proof}

\section{Projections onto linear subspaces}

\begin{theorem}[definitions of a projection onto a subspace]
  Let $U$ be a Hilbert space, $V \le U$.
  Then $V$ is a closed convex set, and so satisfies the previous theorem.
  Let $x \in U$, $y_0 \in V$.
  The following are equivalent:
  \begin{enumerate}
    \item $\n{x-y_0} = \dist\p{X, V}$.
    \item $x-y_0 \perp V$.
  \end{enumerate}  
\end{theorem}

\begin{proof}[$2 \Rightarrow 1$]
  Let $y \in V$.
  Then
  \begin{align*}
    \n{x-y}^2
    &= \n{\p{x-y_0}-\p{y-y_0}}^2 \\
    &= \n{\p{x-y_0}}^2-\n{\p{y-y_0}}^2 \\
    &\ge \n{x-y_0}^2.
  \end{align*}
\end{proof}

\begin{proof}[$1 \Rightarrow 2$]
  \begin{align*}
    &\n{x -\p{y_0+\gb y}}^2 \\
    &= \n{x-y_0}^2 + \a{\gb}^2 \n{y}^2 - 2 \Re\p{\gb \i{y, x-y_0} }.
  \end{align*}
  Choose $\gb$ such that $\gb \i{y, x-y_0} \in \R$ and $\gb \i{y, x-y_0} < 0$.
  Since parabolas are convex, $y \perp x-y_0$ would lead to a contradiction.
\end{proof}

\begin{lemma}
  In the conditions of the previous theorem, let $P \col U \to V$ map every $x \in U$ into the corresponding $y_0$. Then $P$ is linear.
\end{lemma}

\begin{proof}
  Because
  $$ \ga_1 x_1 - Px_1 + \ga_2 x_2 -P x_2 \perp K. $$
\end{proof}

\section{Riesz's lemma}

\begin{lemma}[Riesz]
  Let $H$ be a Hilbert space.
  The following are equivalent:
  ~\begin{enumerate}
    \item $f$ is a continuous functional on $H$.
    \item Exists $y \in H$ such that
    $$ fx = \i{y, x}. $$ 
  \end{enumerate}
\end{lemma}

\begin{proof}[$2 \Rightarrow 1$]
  Clearly, $f$ is a continuous linear functional.
  By CBS, $\n{fx} \le \n{x} \n{y}$.
  This is sharp when $x = y$.
\end{proof}

\begin{proof}[$1 \Rightarrow 2$]
  Let $N := \ker f$.
  If $N = H$, then $f = 0$.
  Suppose $w \in H \sm N$.
  Define $v := w - P_N w$.
  Obviously, $v \perp N$.
  Define $\Phi_v(x) = \i{v, x}$.
  Then $\ker \Phi_v = \ker f$, and so exists $c \in K$ such that $\Phi_v = c f$. Then
  $$ fx = \i{v/c, x}. $$
\end{proof}

\begin{definition}
  A topological vector space $V$ is \emph{reflexive}, iff the canonical map $V \to V^{**}$ is an isomorphism.
\end{definition}

\begin{example}
  The map $\Phi_\s \col H \to H^*$ we have built in the proof is conjugate-linear:
  $$ \Phi_{\ga v} = \ol{\ga} \Phi_v. $$
  Thus any Hilbert space is reflexive.
\end{example}

\section{Orthonormal bases}

\begin{definition}
  A subset $S \ss U$ of a Hilbert space $U$ is \emph{orthonormal}, iff $\n{s} = 1$ and $s \perp s'$ for all different $s, s' \in S$. $S$ is a \emph{basis}, iff it is maximal among orthonormal sets.
\end{definition}

\begin{lemma}
  An orthonormal basis $B$ exists.
\end{lemma}

\begin{proof}
  From Zorn's lemma.
\end{proof}