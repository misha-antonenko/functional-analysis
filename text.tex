\section{Introduction}

Let $X$ be a vector space over a field $K \in \b{\R, \C}$.

\begin{definition}
  A map $f \col X \to \R$ is called a \emph{norm}, iff we have
  ~\begin{enumerate}
    \item If $x \in X \sm 0$, then $f(x) > 0$; and $f(0) = 0$.
    \item $f(kx) = \a{k} f(x)$ for all $k \in \R$, $x \in X$.
    \item $f(x+y) \le f(x)+f(y)$ for all $x, y \in X$.
  \end{enumerate}
\end{definition}

\begin{definition}
  A pair $\p{V, \n{\square}}$, where $\n{\square}$ is a norm on a vector space $V$, is called a \emph{normed space}.
  A normed space is \emph{Banach}, iff it is complete.
\end{definition}

\section{Completions}

\begin{definition}
  $\wh X$ is called a \emph{completion} of $X$, iff $\wh X$ is complete and $X$ is dense in $\wh X$.
\end{definition}

\begin{theorem}
  A completion $\wh X$ exists.
\end{theorem}

\begin{proof}
  Call two Cauchy sequences $\b{x_n}$ and $\b{y_n}$ equivalent, iff $\n{x_n-y_n} \to 0$, and let $\wh X$ be the resulting quotient.
  Since the point-wise sum of two Cauchy sequences is Cauchy, in this natural way we may introduce vector space structure on $\wh X$.
  The norm on $\wh X$ is introduced as
  $$ \n{\q{x_n}} := \lim_{n \to \infty} \n{x_n}. $$
  This map is defined correctly:
  \begin{enumerate}
    \item The limit on the right always exists: since $\b{x_n}$ is Cauchy, the sequence $\b{\n{x_n}}$ of reals is Cauchy, which implies it must converge.
    \item If $[x_n] = [y_n]$, then $\a{\n{x_n}-\n{y_n}} \le \n{x_n-y_n} \xrightarrow[n \to \infty]{} 0.$ Therefore, $\n{[x_n]} = \n{[y_n]}$.
  \end{enumerate}
  $X$ is embedded into $\wh X$ by mapping $x \in X$ into the class of the constant sequence at $x$.
  It is easy to see that this map preserves norms.
\end{proof}

\begin{theorem}
  Let $\wh X_2$ be another completion of $X$.
  Then exists a bijection $f \col \wh X \to \wh X_2$ which is linear, preserves norms,
  and maps the embedded $X$ into the embedded $X$.   
\end{theorem}

These two theorems endow us with the right to never consider pathological incomplete spaces.

\begin{proof}
  Map $[x_n] \in \wh X$ into $\lim x_n \in \wh X_2$.
\end{proof}

\begin{exercise}
  The space $C\q{a, b}$ with the norm
  $\n{f}_2 = \int \a{f}$ is not complete.
  Define
  $$
  f_n(x) = \begin{cases}
    0, & x \in [a, c], \\
    n(x-c), & x \in \q{c, c+ 1/n}, \\
    1, & x \in \q{c+1/n, b}.
  \end{cases}
  $$
  Then $f_n$ is a Cauchy sequence which does not have a limit in $C\q{a, b}$.
\end{exercise}

\begin{proof}
  It is easy to see that the limit of $\b{f_n}$ is $[x \ge c]|_{[a, b]}$, so it does not have a continuous limit. Nevertheless, it is Cauchy, since
  \begin{align*}
    \n{f_n - f_m}
    = \frac{1}{2} \a{\frac{1}{n} - \frac{1}{m}}
    \xrightarrow[\min\b{m,n}\to \infty]{} 0.
  \end{align*}
\end{proof}

\section{Equivalent norms}

\begin{definition}
  We say norms are \emph{equivalent}, iff the metrics they generate are Lipschitz equivalent.
\end{definition}

\begin{exercise}
  Norms are equivalent iff they generate the same topology.
\end{exercise}

\begin{exercise}
  In infinite-dimensional spaces, there are norms which are not equivalent.
\end{exercise}

\begin{proof}
  For example, consider $X = C\q{0, 1}$, $L^1$-norm and the sup-norm on it.
  It is true that $L^1$-norm does not surpass the sup-norm, but there is no constant for the opposite inequality: we can think of a function with an arbitrarily large sup-norm, but constant integral.
\end{proof}

\begin{theorem}
  If $X$ is finite-dimensional, then every two norms on $X$ are equivalent.
\end{theorem}

\begin{proof}
  Suppose $\dim X = n$, and $e_1, \dots, e_n$ is a basis.
  Let $x = a_1 e_1 + \dots + a_n e_n$.
  Let $\n{\square}$ be a norm on $X$.
  Define a new \emph{norm} as
  $$ \a{x} = \sqrt{a_1^2 + \dots + a_n^2}. $$
  Then
  \begin{align*}
    \n{x}
    & \le \sum_{i = 1}^n \a{a_i} \n{e_i} \\
    & \le M \sum_{i=1}^n \a{a_i} \\
    & \le M \p{\sum_{i=1}^n \a{a_i}}^\frac{1}{2} \\
    & = M \a{x}.
  \end{align*}
  
  The function $x \mapsto \n{x}$ is continuous in the norm $\a{\square}$.
  Let $\a{x_k - x} \to 0$.
  Then
  \begin{align*}
    \a{\n{x_k}-\n{x}}
    & \le \n{x_k-x} \\
    & \le M \sqrt{n} \a{x_k-x} \\
    & \to 0.
  \end{align*}
  
  Consider the set $S = \b{x \in E \mid \a{x} = 1}$.
  $S$ is compact in $\a{\square}$.
  $\phi|_S$ is continuous and nonzero.
  Then $\phi > \gd$ for some $\gd > 0$, so
  $$ \n{\frac{x}{\a{x}}} \ge \gd \iff \n{x} \ge \gd \a{x}. $$ 
\end{proof}

\begin{corollary}
  Every finite-dimensional normed vector space is complete.
\end{corollary}

\begin{proof}
  Every Euclidean space is complete.
\end{proof}

\begin{corollary}
  \label{finite-dimensional implies closed}
  A finite-dimensional subspace of a normed space is closed.
\end{corollary}

\begin{proof}
  It is complete, and every convergent sequence is Cauchy sequence.
\end{proof}

\begin{definition}
  The set $M \ss X$ is \emph{bounded}, iff
  $$ \sup_{m \in M} \n{m} < +\infty. $$
\end{definition}

\begin{lemma}[on an almost-perpendicular]
  Let $E$ be a normed space, and $F < E$ its closed proper subspace.
  Then for every $\eps > 0$ exists a vector $x \in E$ such that $\n{x} = 1$ and $\dist(x, F) > 1-\eps$.
\end{lemma}

\begin{proof}
  Let $y \in E \sm F$.
  Then $d = \on{dist}\p{y, F} > 0,$ since $F$ is closed.
  Let $\gd > 0$.
  By definition of infimum, there exists $a \in F$ such that
  $$ d \le \n{y - a} \le d + \gd. $$
  Put $y_2 = y-a$.
  Since $a \in F$, $\on{dist}\p{y_2, F} = \dist(y, F) = d$.
  Define
  $$ x = \frac{y_2}{d+\gd} = \frac{y-a}{d+\gd}. $$
  Then $\n{x} \le 1$, but
  $$ \on{dist}\p{x, F} = \dist\p{\frac{y}{d+\gd}, F} \ge \frac{\dist(y, F)}{d+\gd} = \frac{d}{d+\gd}. $$
  Since the $\gd$ is arbitrary, and by increasing the norm of $x$ we do not get closer to $F$, we get the desired.
\end{proof}

\begin{theorem}
  Let $X$ be a normed space. Equivalent are:
  \begin{enumerate}
    \item $X$ is finite-dimensional.
    \item Every bounded subset of $X$ is relatively compact.
  \end{enumerate}  
\end{theorem}

\begin{proof}[Proof of $1 \implies 2$.]
  From corollary on page \pageref{finite-dimensional implies closed}.
\end{proof}

\begin{proof}[Proof of $2 \implies 1$.]
  Suppose $X$ is not finite-dimensional.
  We assert that there exists a bounded sequence that does not have a convergent subsequence (so in no way the closure of a bounded subset that contains this sequence can be compact).
  We show this by induction: suppose $x_1, \dots, x_n$ are already built.
  By the almost-perpendicular lemma there exists $x_{n+1}$ such that
  $\n{x_{n+1}} = 1$ and
  $$ \dist\p{x_{n+1}, \Span\b{x_1, \dots, x_n}} > {1}/{2} $$
  (since $X$ is not finite-dimensional, the span here is a proper subspace of $X$).
  Continuing to infinity, we get a sequence $\b{x_n}$. It is bounded (all its members are on the unit sphere). Nevertheless, no subsequence of it is Cauchy by construction.
\end{proof}

\section{Linear operators}

\begin{definition}
  A linear operator $T \col X \to Y$ is \emph{bounded}, iff $T(B)$ is bounded, where $B$ is the unit ball in $X$.
\end{definition}

\begin{lemma}
  Let $X$ and $Y$ be normed vector spaces, and $T \col X \to Y$ a linear operator.
  The following are equivalent:
  \begin{enumerate}
    \item $T$ is continuous.
    \item $T$ is continuous at 0.
    \item $T$ is bounded.
  \end{enumerate}
\end{lemma}

\begin{proof}[Proof of $1 \Leftrightarrow 2$]
  Let $x \in X$. $T$ is continuous at $x$ iff for every convergent $x_n \to x$ the sequence $Tx_n$ also converges (to $Tx$).
  Now observe that
  \begin{align*}
    \n{Tx_n-Tx}
    &= \n{T\p{x_n-x}}.
  \end{align*}
\end{proof}

\begin{proof}[Proof of $2 \Rightarrow 3$]
  Consider $$ D = \b{y \in Y \mid \n{y} \le 1}. $$
  There is $\gd > 0$ such that, if $\n{x} \le \gd$, then $Tx \in D$.
  Let $z \in D$.
  Since
  $$\n{T{\gd z}} \le 1,$$ we have
  $$ \n{Tz} \le {1}/{\gd}. $$
\end{proof}

\begin{proof}[Proof of $3 \Rightarrow 1$]
  Let $\n{x} \le 1$. Then $\n{Tx} \le M$, so $\n{x} < \eps$ implies $\n{Tx} \le M \eps$.
\end{proof}

\subsection{Operator norm}

\begin{definition}
  Let $T$ is a continuous operator. Its \emph{norm} is
  $$ \n{T} = \sup_{x \ne 0} \frac{\n{Tx}}{\n{x}}. $$
\end{definition}

\begin{exercise}
  $\n{T}$ is indeed a norm.
\end{exercise}

\begin{lemma}
  An operator is bounded iff it has finite norm.
\end{lemma}

\begin{proof}
  Obvious.
\end{proof}

\begin{lemma}
  A linear combination of continuous operators is continuous.
\end{lemma}

\begin{proof}
  \begin{align*}
    \n{(kA+B)} \le \a{k}\n{A}+\n{B}.
  \end{align*}
\end{proof}

\subsection{The space of bounded operators is complete}

\begin{definition}
  $\mc{B}\p{X, Y}$ is the set of bounded linear operators $X \to Y$ with the obvious structure of a vector space and the standard operator norm as the norm.
\end{definition}

\begin{theorem}
  Let $Y$ be complete.
  Then $\mc{B}\p{X, Y}$ is complete.  
\end{theorem}

\renewcommand{\B}[2]{\mc{B}\p{{#1}, {#2}}}

{\footnotesize I have seen three quite concise proofs of this theorem and understood neither. Here is a long one, my own.}

\begin{proof}
  Let $\b{T_n}$ be a Cauchy sequence.
  Fix $x \in X$.
  $\b{T_n x}$ is a Cauchy sequence in $Y$:
  \begin{align*}
    \n{T_nx - T_mx}
    &\le \n{T_n-T_m}\n{x} \\
    &\le \eps \n{x}.
  \end{align*}
  By completeness of $Y$, there is a limit $t \xleftarrow[n \to \infty]{} T_n x$.
  
  The map $x \mapsto t$ we have just build is a linear operator. Call it $T$.
  $T_n$ is a Cauchy sequence, so $\n{T_n} \le M$ for some $M$.
  Then $T$ itself is bounded:
  \begin{align*}
    \n{Tx}
    &\le \n{T_nx}+ \eps\\
    &\le M \n{x} + \eps.
  \end{align*}
  
  We assert that $\n{T_nx-Tx} \to 0$.
  Suppose otherwise:
  \begin{align*}
    \ex \eps > 0\ \fa n_0 \in \N\ \ex n > n_0\ \ex x \in B \col \n{T_n x - Tx} > \eps.
  \end{align*}
  Since $\b{T_n}$ is Cauchy,
  \begin{align*}
    \fa \gd > 0\ \ex n_1 \in \N\ \fa k, l > n_1\ \fa x \in B \col \n{T_kx-T_lx} \le \gd.
  \end{align*}
  Fix this $\gd > 0$ and take the corresponding $n_1$.
  From the first line with quantifiers, there exist $n > n_1$ and $x \in B$ such that
  $$ \n{T_n x - Tx} > \eps. $$
  From the second one, for any $m > n_1$ we get
  \begin{align*}
    \n{T_n x - Tx}
    &\le \n{T_nx - T_mx} + \n{T_mx-Tx} \\
    &\le \gd + \n{T_mx-Tx}.
  \end{align*}
  Since $Tx = \lim_{m \to \infty} T_mx$, there is $m_0$ such that $\n{T_mx-Tx} \le \gd$ for all $m > m_0$.
  Take $m_1 = \max\b{m_0, n_1}$. Then, for any $m > m_1$,
  \begin{align*}
    \n{T_n x - Tx}
    &\le \gd + \n{T_mx-Tx} \\
    &\le 2\gd.
  \end{align*}
  Now launch $n_1 \to \infty$ and, consequently, $\gd \to 0$. We get a contradiction:
  \begin{align*}
    \eps < \n{T_n x - Tx} \le 2\gd.
  \end{align*}
\end{proof}

\begin{remark}
  Our proof does not use the boundedness of operators in the space $\mathcal{B}(X, Y)$.
\end{remark}

\section{Functionals}

\begin{definition}
  A \emph{functional} is a linear operator $X \to K$.
\end{definition}

\begin{definition}
  The \emph{dual} $X^*$ of $X$ is the space $\mathcal{B}(X, K)$ of continuous functionals.
\end{definition}

\begin{corollary}
  $X^*$ is complete.
\end{corollary}

\begin{proof}
  Since $K$ is complete in either case.
\end{proof}

\section{Strong convergence}

\begin{definition}
  A sequence of operators $T_n \to T$ converges \emph{strongly} or \emph{point-wise}, iff $T_n x \to Tx$ for all $x \in X$. 
\end{definition}

\begin{proposition}
  If $\n{T_n - T} \to 0$, then $T_n \to T$ strongly.
\end{proposition}

\begin{proof}
  Since
  \begin{align*}
    \a{T_nx - Tx}
    &\le \n{T_n-T}\a{x}.
  \end{align*}
\end{proof}

\begin{remark}
  The converse is not true.  
\end{remark}

\begin{definition}
  $l^p = L^p(\N, \#)$ is the space of real-valued sequences which converge in the $L^p$ norm (with respect to the cardinality measure). 
\end{definition}

\begin{proof}[Proof of the remark]
  Consider the operators
  \begin{align*}
    s_k (c_0, c_1, \dots) = (c_k, c_{k+1}, \dots)
  \end{align*}
  on $l^p$. $\n{s_k} = 1$, since there are sequences with a single unit and other elements zero and applying $s_k$ does not lessen the sequence norm anyway.
  Nevertheless, $s_k \to 0$ pointwise (strongly), since all the sequences in $l^p$ converge.
\end{proof}

\section{More on Banach spaces}

\begin{theorem}
  Let $Y$ be complete.
  Let $\b{T_n} \ss \mc{B}(X, Y)$ be operators with $\sup \n{T_n} < +\infty$, and let $E \ss X$ be dense.
  Suppose that, for every $e \in E$, the sequence $\b{T_ne} \ss Y$ converges.
  Then exists $T \in \mc{B}(X, Y)$ such that $T_n \to T$ pointwise.
\end{theorem}

\begin{remark}
  If $\b{T_n}$ converges strongly, then $\sup \n{T_n} < +\infty$.
  This is a harder fact.  
\end{remark}

\begin{proof}
  Define
  $$ Te := \lim_{n \to \infty} T_n e. $$
  Let $x\in x$. Take $\b{e_n}$ such that $\n{e_n-x} \xrightarrow[n\to\infty]{} 0$.
  We assert $T$ can be continued to a bounded operator $X \to Y$.
  
  $\b{Te_n}$ is Cauchy.
  \begin{align*}
    \n{Te_k-Te_k}
    &\le \n{Te_j-T_ne_j} + \n{T_ne_j-T_ne_k}+\n{Te_k-T_ne_k} \\
    &< 2\eps + \n{T_ne_j-T_ne_k} \\
    &\le 2\eps + \n{T_n}\n{e_j-e_k} \\
    &\xrightarrow[\substack{n\to \infty\\ \min\b{j,k}\to\infty}]{} 0.
  \end{align*}
\end{proof}

\begin{theorem}
  Let $X$ and $Y$ be normed spaces, and $Y$ complete.
  Let $F$ be dense in $X$.
  Let $T \col F \to Y$ be a continuous linear operator.
  Then exists unique $T_\sharp  \in \mc{B}(X, Y)$ such that $\n{T_\sharp} = \n{T}$ and $T_\sharp|_F = T$. 
\end{theorem}

\begin{proof}
  Let $x \in X$, $f_n \in F$, $\n{f_n-x} \to 0$.
  Then
  \begin{align*}
    \n{Tf_n - Tf_m}
    &\le \n{T} \n{f_n-f_m} \\
    &\xrightarrow[\min\b{m,n} \to \infty] 0.
  \end{align*}
  Hence $\b{Tf_n}$ is a Cauchy sequence and, as such, converges to some $y$.
  Define
  $$ T_\sharp x = y. $$
  This definition does not depend on the choice of the sequence.
\end{proof}

\section{Three fundamental principles of functional analysis}

\begin{enumerate}
  \item Hanh-Banach theorem.
  \item Closed graph and open map theorems.
  \item The principle of uniform boundedness.
\end{enumerate}


\section{Hanh-Banach theorem}

\begin{definition}
  Let $X$ be a vector space over $K$, $p \col X \to \R$.
  It is said that $p$ is a \emph{seminorm}, iff
  \begin{enumerate}
    \item $p(x) \ge 0$.
    \item $\a{k}p(x) = p(kx)$ for all $k \in K$.
    \item $p(x+y) \le p(x)+p(x)$.
  \end{enumerate}
\end{definition}

\begin{example}
  Consider $X = C\p{\R}$ and $p(f) = \int_0^1 \a{f}$. Then $p$ is a seminorm, but not a norm, since there is a nonzero function, which is zero on $[0, 1]$.
\end{example}

\begin{definition}
  Let $X$ be a $K$-vector space. A $p \col X \to \R$ is a \emph{sublinear functional}, iff
  \begin{enumerate}
    \item $p(x+y) \le p(x)+p(y)$.
    \item $p(kx) = \a{k}p(x)$ for all $k \in K$.
  \end{enumerate}
\end{definition}

\begin{theorem}[Hanh, Banach]
  Let $X$ be a real vector space, $p \col X \to \R$ a sublinear functional.
  Let $Y \le X$ and $f \col Y \to \R$ a linear functional such that $f(y) \le p(y)$ for all $y \in Y$.
  Then exists $F \col X \to \R$ such that $F|_Y = f$ and $F(x) \le p(x)$ for all $x \in X$.
\end{theorem}

\begin{lemma}[Hanh-Banach in codimesion 1]
  Ler $x_0 \in X \sm Y$.
  Let $Y_\sharp = Y + \Span\b{y_0}$.
  Then $f$ can be continued to a linear functional on $f_\sharp  \col Y_\sharp \to \R$ such that $f_\sharp \le p$ on $Y_\sharp$.
\end{lemma}

\begin{proof}
  Let $y \in Y_\sharp$, $y \in Y$, $\ga \in \R$.
  
  We assert a $c \in \R$ can be chosen in such a way that
  $$ f_\sharp(y+\ga y_0) := f(y) + \ga c $$
  satisfies
  \begin{equation}
    \label{j1}
    f_\sharp \le p \iff f(y)+\ga c \le p\p{y+\ga y_0}.
  \end{equation}
  
  If $\ga = 0$, the inequality is satisfied.
  
  Suppose $\ga > 0$.
  Divide \eqref{j1} through by $\ga$:
  $$ f\p{\frac{y}{\ga}} + c \le p\p{\frac{y}{\ga}+y_0}. $$
  This rewrites as
  $$ p\p{y_1+y_0} - f\p{y_1} \ge c, $$
  where $y_1= {y}/{\ga}$.
  
  Suppose $\ga < 0$. Dividing \eqref{j1} through by $-\ga$, we get
  $$ f\p{y_2}-p\p{y_2-y_0} \le c, $$
  where $y_2 = -y/\ga$.
  
  Now, if we show that
  $$ f\p{y_2}-p\p{y_2-y_0} \le p\p{y_1+y_0} - f\p{y_1}, $$
  we are done by the Cantor-Dedekind axiom. But that trivially follows from the triangle inequality for $p$ and linearity of $f$.
\end{proof}

\begin{proof}[Proof of the theorem of Hanh and Banach in the general case]
  Consider the set
  $$ A = \b{(M, f_M) \mid Y \le M \le X,\ f_M \col M \to \R \text{ is a linear functional},\ f_M \le p}. $$
  We tell that $(M, f_M) \le (N, f_N)$, iff $M \le N$ and $f_N|_M = f_M$.
  The union of any chain is its supremum; we have a maximal element $(L, F)$.
  We assert that $L = X$.
  
  Suppose otherwise, and let $y_0 \in X \sm L$.
  Define $L_\sharp = L + \Span\b{y_0}$.
  By the lemma, there exists $F_\sharp \col L_\sharp \to \R$ such that $F_\sharp \le p.$
  But then $(L_\sharp, F_\sharp) \ge (L, F)$. 
\end{proof}

\newcommand{\cB}[0]{\mc{B}}

\subsection{Useful corollaries}

\begin{corollary}
  Let $Y \le X$ and $f \in \mc{B}(Y, \R)$. Then there exists $F \col X \to \R$ such that $F|_Y = f$ and $\n{F} = \n{f}$.
\end{corollary}

\begin{proof}
  Let $p(x) = \n{f} \n{x}$. Then $f(y) \le p(y)$ for all $y \in Y$.
  Take $F$ as in the HB theorem.
  Then
  $$ F(x) \le \n{f} \n{x}. $$
  Likewise, taking the $-F$ for $-f$, we get
  $$ \n{F} \le \n{f}. $$
  The converse inequality is evident.
\end{proof}

\begin{corollary}
  Let $Y \le X$, $x_0 \in X \sm Y$. Then exists $F \in X^*$ such that $\n{F} \le 1$, $F|_Y = 0$, and $F\p{x_0} = \dist\p{x_0, Y}$.
\end{corollary}

\begin{proof}
  Define $p(x) = \dist(x, Y)$.
  Since $Y$ is closed, $d := p(x_0) > 0$.
  Define
  $$ f(y+\ga x_0) := \ga d. $$
  Then $f \le p$, and exists $F \col X \to \R$ such that $F|_L = f|_L$ and $F \le p$.
  In particular, $F|_Y = 0$ (what we need) and $F(x_0) = d$.
  Observe that the same applies to $-f$, and so $\n{F(x)} \le p(x) \le \n{x}$.
  Hence $\n{F} \le 1$.
\end{proof}

\begin{corollary}
  Let $x_0 \in X$. Then exists $F \in X^*$ such that $\n{F(x_0)} = \n{x_0}$ and $\n{F} = 1$.
\end{corollary}

\begin{proof}
  The case $Y = \b{0}$ of the previous corollary.
\end{proof}

\section{Banach limits}

\begin{definition}
  Let $c \le l^\infty$ be the space of convergent subsequences.
  A map $L \col l^\infty \to \R$ is a \emph{Banach limit}, iff
  \begin{enumerate}
    \item It is a continuous linear functional.
    \item For all $x = \b{x_n} \in c$, $$ Lx = \lim x_n. $$
    \item If $x \ge 0$, then $Lx \ge 0$.
    \item $L\b{x_{n+1}} = L\b{x_n}$.
  \end{enumerate}
\end{definition}

\begin{theorem}
  $L$ exists.
\end{theorem}

\begin{proof}
  Put, for $x = \b{x_n}$,
  $$ f(x) = \lim x_n. $$
  $f \col c \to \R$ is a linear functional.
  Define a majoring sublinear functional (exercise) as
  $$ p(x) = \limsup \frac{1}{n} \sum_{i=1}^n x_i. $$
  Observe that $p|_c = f$, so $f \le p$.
  By HB, we have $L \col l^\infty \to \R$, which, luckily, satisfies the definition of Banach limit:
  \begin{enumerate}
    \item $L|_c = f$.
    \item If $x \le 0$, then $Lx \le p(x) \le 0$.
    \item Put $x'_i = x_{i+1}-x_i$. By linearity of $L$, it is sufficient to show that $L\b{x'_n} = 0$. And indeed,
      $$ p(x') = \lim \frac{1}{n} \sum_{i=1}^n \p{x_{n+1}-x_n} = 0. $$
  \end{enumerate}
\end{proof}

\subsection{The complex case}

\begin{definition}
  Let $c \le l^\infty(\C)$ be the space of convergent subsequences.
  A map $L \col l^\infty \to \C$ is a \emph{Banach limit}, iff
  \begin{enumerate}
    \item It is a continuous linear functional.
    \item For all $x = \b{x_n} \in c$, $$ Lx = \lim x_n. $$
    \item If $x \in \R$ and $x \ge 0$, then $Lx \ge 0$.
    \item $L\b{x_{n+1}} = L\b{x_n}$.
    \item $\n{L} = 1$.
  \end{enumerate}
\end{definition}

\begin{proof}
  Let $L$ be a real Banach limit.
  For $a, b \in l^\infty(\C)$, define
  $$ L(a+ib) := La + iLb. $$
  All of the properties now follow trivially from those of real limits, except the final one.
  
  Simple functions are dense in $l^\infty$, so we may prove the statement for them and be happy after using continuity of $L$.
  Let $x = \sum \ga_k \chi_{E_k}$ for some partition $\bigsqcup_{k \in \N} E_k = \N$, and $\a{\ga_\s} \le 1$.
  Then $Lx = \sum \ga_k L\chi_{E_k}$, and
  \begin{align*}
    \a{Lx}
    &\le \sum_{k \in \N} L\p{\chi_{E_k}} \\
    &= L\p{\chi_{\sqcup E_k}} \\
    &\le 1,
  \end{align*}
  which was asserted.
\end{proof}